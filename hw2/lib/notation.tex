% Name: notation.tex
% Copyright: %W% (C) Henry Thomas.	Release %R%	Dated %G%
% Version: %R%
% Purpose: Definition file for notation
% Comment: Relies on Piet Tutelaers macros for chess typesetting
% Author: Henry Thomas
% Date: %G%
%
%% Some symbols
\newcommand{\FigDash}{{--}}
\newcommand{\FigCapt}{$\times$}
\newcommand{\FigDots}{$\ldots$}
\newcommand{\FigDot}{\char46}
%
% Redefine chess.sty to get similar layout as in Informator
\makeatletter
\gdef\@dots{\ldots}
\resetat
%
% These macros allows a move number to be bolfaced in the main line,
% and typset in roman in the variations (like in Informant)
\newcommand{\mn}[1]{{\OutLine #1\char46}}
%
% Some macros to improve readability...
\newcommand{\OutLine}{\bf}
\newenvironment{Mainline}[2]{\newcommand{\result}{#1}%
   \newcommand{\commentator}{#2}\begin{chess}}%
   {\end{chess}\finito{\result}{\commentator}}
\newenvironment{Variation}{{\OutLine
[}\begingroup\renewcommand{\OutLine}{\rm}\ignorespaces}%
   {\endgroup{\OutLine ]}\renewcommand{\OutLine}{\bf}\ignorespaces}
\newenvironment{diagram}{\begin{nochess}}{$$\showboard$$\end{nochess}}
\newcommand{\finito}[2]{{\bf\hfill#1\hfill[#2]\par}}

\setlength{\parindent}{0pt}

% If you compiled notation with the flag -DOLDTEX, you will need to
% define these macros.
% They allow to typeset in horizontal or vertical mode. You ``simply''
% have to define them.
%
\newcommand{\MoveNumber}[1]{{#1}~}
\newcommand{\WhiteMove}[1]{{#1}\ }
\newcommand{\BlackMove}[1]{{#1}\ }
%
% These macros are for tiles and score formatting.
\newcommand{\ChessTitle}[1]{\begin{center}{\large\bf #1}\end{center}}
\newcommand{\ChessSubTitle}[1]{\begin{center}{\sc #1}\end{center}}
\newcommand{\ChessScore}[1]{{\bf #1}}
%
